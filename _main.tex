% Options for packages loaded elsewhere
\PassOptionsToPackage{unicode}{hyperref}
\PassOptionsToPackage{hyphens}{url}
%
\documentclass[
]{book}
\usepackage{lmodern}
\usepackage{amsmath}
\usepackage{ifxetex,ifluatex}
\ifnum 0\ifxetex 1\fi\ifluatex 1\fi=0 % if pdftex
  \usepackage[T1]{fontenc}
  \usepackage[utf8]{inputenc}
  \usepackage{textcomp} % provide euro and other symbols
  \usepackage{amssymb}
\else % if luatex or xetex
  \usepackage{unicode-math}
  \defaultfontfeatures{Scale=MatchLowercase}
  \defaultfontfeatures[\rmfamily]{Ligatures=TeX,Scale=1}
\fi
% Use upquote if available, for straight quotes in verbatim environments
\IfFileExists{upquote.sty}{\usepackage{upquote}}{}
\IfFileExists{microtype.sty}{% use microtype if available
  \usepackage[]{microtype}
  \UseMicrotypeSet[protrusion]{basicmath} % disable protrusion for tt fonts
}{}
\makeatletter
\@ifundefined{KOMAClassName}{% if non-KOMA class
  \IfFileExists{parskip.sty}{%
    \usepackage{parskip}
  }{% else
    \setlength{\parindent}{0pt}
    \setlength{\parskip}{6pt plus 2pt minus 1pt}}
}{% if KOMA class
  \KOMAoptions{parskip=half}}
\makeatother
\usepackage{xcolor}
\IfFileExists{xurl.sty}{\usepackage{xurl}}{} % add URL line breaks if available
\IfFileExists{bookmark.sty}{\usepackage{bookmark}}{\usepackage{hyperref}}
\hypersetup{
  pdftitle={Szövegbányászat és mesterséges intelligencia R-ben},
  pdfauthor={Sebők Miklós, Ring Orsolya},
  hidelinks,
  pdfcreator={LaTeX via pandoc}}
\urlstyle{same} % disable monospaced font for URLs
\usepackage{color}
\usepackage{fancyvrb}
\newcommand{\VerbBar}{|}
\newcommand{\VERB}{\Verb[commandchars=\\\{\}]}
\DefineVerbatimEnvironment{Highlighting}{Verbatim}{commandchars=\\\{\}}
% Add ',fontsize=\small' for more characters per line
\usepackage{framed}
\definecolor{shadecolor}{RGB}{248,248,248}
\newenvironment{Shaded}{\begin{snugshade}}{\end{snugshade}}
\newcommand{\AlertTok}[1]{\textcolor[rgb]{0.94,0.16,0.16}{#1}}
\newcommand{\AnnotationTok}[1]{\textcolor[rgb]{0.56,0.35,0.01}{\textbf{\textit{#1}}}}
\newcommand{\AttributeTok}[1]{\textcolor[rgb]{0.77,0.63,0.00}{#1}}
\newcommand{\BaseNTok}[1]{\textcolor[rgb]{0.00,0.00,0.81}{#1}}
\newcommand{\BuiltInTok}[1]{#1}
\newcommand{\CharTok}[1]{\textcolor[rgb]{0.31,0.60,0.02}{#1}}
\newcommand{\CommentTok}[1]{\textcolor[rgb]{0.56,0.35,0.01}{\textit{#1}}}
\newcommand{\CommentVarTok}[1]{\textcolor[rgb]{0.56,0.35,0.01}{\textbf{\textit{#1}}}}
\newcommand{\ConstantTok}[1]{\textcolor[rgb]{0.00,0.00,0.00}{#1}}
\newcommand{\ControlFlowTok}[1]{\textcolor[rgb]{0.13,0.29,0.53}{\textbf{#1}}}
\newcommand{\DataTypeTok}[1]{\textcolor[rgb]{0.13,0.29,0.53}{#1}}
\newcommand{\DecValTok}[1]{\textcolor[rgb]{0.00,0.00,0.81}{#1}}
\newcommand{\DocumentationTok}[1]{\textcolor[rgb]{0.56,0.35,0.01}{\textbf{\textit{#1}}}}
\newcommand{\ErrorTok}[1]{\textcolor[rgb]{0.64,0.00,0.00}{\textbf{#1}}}
\newcommand{\ExtensionTok}[1]{#1}
\newcommand{\FloatTok}[1]{\textcolor[rgb]{0.00,0.00,0.81}{#1}}
\newcommand{\FunctionTok}[1]{\textcolor[rgb]{0.00,0.00,0.00}{#1}}
\newcommand{\ImportTok}[1]{#1}
\newcommand{\InformationTok}[1]{\textcolor[rgb]{0.56,0.35,0.01}{\textbf{\textit{#1}}}}
\newcommand{\KeywordTok}[1]{\textcolor[rgb]{0.13,0.29,0.53}{\textbf{#1}}}
\newcommand{\NormalTok}[1]{#1}
\newcommand{\OperatorTok}[1]{\textcolor[rgb]{0.81,0.36,0.00}{\textbf{#1}}}
\newcommand{\OtherTok}[1]{\textcolor[rgb]{0.56,0.35,0.01}{#1}}
\newcommand{\PreprocessorTok}[1]{\textcolor[rgb]{0.56,0.35,0.01}{\textit{#1}}}
\newcommand{\RegionMarkerTok}[1]{#1}
\newcommand{\SpecialCharTok}[1]{\textcolor[rgb]{0.00,0.00,0.00}{#1}}
\newcommand{\SpecialStringTok}[1]{\textcolor[rgb]{0.31,0.60,0.02}{#1}}
\newcommand{\StringTok}[1]{\textcolor[rgb]{0.31,0.60,0.02}{#1}}
\newcommand{\VariableTok}[1]{\textcolor[rgb]{0.00,0.00,0.00}{#1}}
\newcommand{\VerbatimStringTok}[1]{\textcolor[rgb]{0.31,0.60,0.02}{#1}}
\newcommand{\WarningTok}[1]{\textcolor[rgb]{0.56,0.35,0.01}{\textbf{\textit{#1}}}}
\usepackage{graphicx}
\makeatletter
\def\maxwidth{\ifdim\Gin@nat@width>\linewidth\linewidth\else\Gin@nat@width\fi}
\def\maxheight{\ifdim\Gin@nat@height>\textheight\textheight\else\Gin@nat@height\fi}
\makeatother
% Scale images if necessary, so that they will not overflow the page
% margins by default, and it is still possible to overwrite the defaults
% using explicit options in \includegraphics[width, height, ...]{}
\setkeys{Gin}{width=\maxwidth,height=\maxheight,keepaspectratio}
% Set default figure placement to htbp
\makeatletter
\def\fps@figure{htbp}
\makeatother
\setlength{\emergencystretch}{3em} % prevent overfull lines
\providecommand{\tightlist}{%
  \setlength{\itemsep}{0pt}\setlength{\parskip}{0pt}}
\setcounter{secnumdepth}{-\maxdimen} % remove section numbering
\ifluatex
  \usepackage{selnolig}  % disable illegal ligatures
\fi

\title{Szövegbányászat és mesterséges intelligencia R-ben}
\author{Sebők Miklós, Ring Orsolya}
\date{}

\begin{document}
\frontmatter
\maketitle

\mainmatter
\hypertarget{bevezetuxe9s}{%
\chapter{Bevezetés}\label{bevezetuxe9s}}

Jelen kötet a Kvantitatív szövegelemzés és szövegbányászat a
politikatudományban (L'Harmattan, 2016) című könyv folytatásaként és
egyben kiegészítéseként a szövegbányászat és a mesterséges intelligencia
társadalomtudományi alkalmazásának gyakorlatába nyújt bevezetést. A
szövegek kvantitatív elemzése (quantitative text analysis -- QTA) a
nemzetközi társadalomtudomány egyik leggyorsabban fejlődő irányzata. A
szövegek és más minőségi adatok (filmek, képek) elemzése annyiban
különbözik a mennyiségi (kvantitatív) adatokétól, hogy nyers formájukban
még nem alkalmasak arra, hogy statisztikai, illetve ökonometriai elemzés
alá vessük őket, s így további módszertani problémákat vetnek fel,
melyek speciális tárgyalása szükséges. A tervezett kötetben bemutatott
példák többsége a politikatudományhoz kapcsolódik, de más alkalmazási
területekre is kitér.

Míg az előző kötet az egyes kódolási eljárásokat, illetve ezek
kutatás-módszertani előnyeit és hátrányait ismertette, itt a
társadalomtudományi elemzésének során használható kvantitatív
szövegelemzés legfontosabb gyakorlati feladatait vesszük sorra. A
kézirat a magyar tankönyvpiacon az elsőnek számít a tekintetben, hogy a
társadalomtudományban használatos kvantitatív szövegelemzési eljárásokat
részletesen, lépésről-lépésre ismerteti, kezdve a megfelelő korpusz
kialakításához szükséges ismeretektől, a különböző szövegbányászati
módszerek (szózsák, dokumentum-kifejezés mátrix, a névelem-felismerés,
az osztályozás, illetve a csoportosítás feladataira), illetve az
egyszerűbb szövegösszehasonlítási-feladatok áttekintésén át, egészen a
felügyelt és felügyelet nélküli gépi tanulásig, a politikatudományi
vizsgálatok során leggyakrabban használatos R programnyelven készült
programok segítségével.

Az olvasó a két kötet együttes használatával olyan ismeretek birtokába
kerül, melyek révén képes lesz alkalmazni a kvantitatív szövegelemzés és
szövegbányászat legalapvetőbb eljárásait saját kutatására. Deduktív vagy
induktív felfedező logikája fényében dönthet az adatelemzés módjáról, és
a felkínált menüből kiválaszthatja a kutatási tervéhez legjobban
illeszkedő megoldásokat. A kötetet végigkísérő konkrét példák
segítségével pedig akár reprodukálhatja is ezen eljárásokat saját
kutatásában. Mindezt a kötet függelékében helyet kapó R-scriptek
részletes leírása is segíti majd. A kötet két fő célcsoportjaként így a
társadalomtudományi kutatói és felsőoktatási közösséget határozzuk meg,
valamint rögzítjük, hogy a kvantitatív szövegelemzés területén belül
elsődlegesen a dokumentum- és tartalomelemzési módszertanhoz
kapcsolódunk.

A könyvben ugyancsak helyet kap a fontosabb fogalmak magyar és angol
nyelvű szószedete, valamint a további olvasásra ajánlott szakirodalom
szerepeltetése. Az oktatásban való közvetlen alkalmazást segíthetik
továbbá a fejezetek végén megadott vizsgakérdések, illetve a kötet
honlapján (qta.tk.mta.hu) szereplő további információk:
gyakorlófeladatok (megoldásokkal), az egyes feladatokra alkalmazható
scriptek és kereskedelmi programok bemutatása, a témával kapcsolatos
prezentációk és további ajánlott irodalmak.

\hypertarget{a-kvantitatuxedv-szuxf6vegelemzuxe9s-uxe9s-szuxf6vegbuxe1nyuxe1szat-alapfogalmai}{%
\chapter{A kvantitatív szövegelemzés és szövegbányászat
alapfogalmai}\label{a-kvantitatuxedv-szuxf6vegelemzuxe9s-uxe9s-szuxf6vegbuxe1nyuxe1szat-alapfogalmai}}

elso fejezet

\hypertarget{az-adatkezeluxe9s-uxe9s-vizualizuxe1ciuxf3-alapjai}{%
\chapter{Az adatkezelés és vizualizáció
alapjai}\label{az-adatkezeluxe9s-uxe9s-vizualizuxe1ciuxf3-alapjai}}

\hypertarget{a-pipe-operuxe1tor}{%
\section{A pipe operátor}\label{a-pipe-operuxe1tor}}

A pipe operátor \texttt{\%\textgreater{}\%} mindig egy parancs kimenetét
veszi fel, és a következő parancs bemenetévé teszi azt, így segítségével
tulajdonképpen folyamat láncolatokat (pipeline) képezhetünk. Mivel az R
egy funkcionális nyelv, az R-ben írt kódok gyakran sok zárójelet
tartalmaznak. Összetett kódok esetén sok zárójelet kellene egymásba
ágyazni, ami megnehezítené a kód olvasását és megértését, ekkor segít a
pipe operator. Az alábbiakban például azt akarjuk összeszámolni, hogy
hány Anna nevű lány született, amit megtehetjük zárójelek egymásba
ágyazásával:

\texttt{sum(select(filter(babynames,sex\ ==\ "F",name\ ==\ "Anna"),n))}

De ugyanezt megkaphatjuk a\texttt{\%\textgreater{}\%} használatával is:

\begin{verbatim}
babynames %>%
filter(sex == "F", name == "Anna") %>%
            select(n) %>%
            sum()
\end{verbatim}

\hypertarget{egy-data-frame-meghatuxe1rozott-sorainak-levuxe1logatuxe1sa}{%
\section{Egy data frame meghatározott sorainak
leválogatása}\label{egy-data-frame-meghatuxe1rozott-sorainak-levuxe1logatuxe1sa}}

A következőkben az előző fejezetben megismert data frame meghatározott
sorainak leválogatásra láthatunk példát a fentebb bemutatott pipe
operátor segítségével. Ehhez az \texttt{install.packages()} paranccsal
installáljuk, majd a \texttt{library()}függvény segítségével olvassuk be
a következő csomagokat:

\begin{Shaded}
\begin{Highlighting}[]
\FunctionTok{install.packages}\NormalTok{(readr)}
\FunctionTok{install.packages}\NormalTok{(tidyr)}
\FunctionTok{install.packages}\NormalTok{(dplyr)}
\FunctionTok{install.packages}\NormalTok{(purrr)}
\FunctionTok{install.packages}\NormalTok{(ggplot2)}
\FunctionTok{install.packages}\NormalTok{(gapminder)}
\end{Highlighting}
\end{Shaded}

\begin{Shaded}
\begin{Highlighting}[]
\FunctionTok{library}\NormalTok{(readr)}
\FunctionTok{library}\NormalTok{(tidyr)}
\FunctionTok{library}\NormalTok{(dplyr)}
\FunctionTok{library}\NormalTok{(purrr)}
\FunctionTok{library}\NormalTok{(ggplot2)}
\FunctionTok{library}\NormalTok{(gapminder)}
\end{Highlighting}
\end{Shaded}

Ezután hozzuk létre a gapminder nevű data frame-t:

\begin{Shaded}
\begin{Highlighting}[]
\NormalTok{gapminder\_df }\OtherTok{\textless{}{-}}\NormalTok{ gapminder}
\end{Highlighting}
\end{Shaded}

A jobb felső ablak environment fülén láthatjuk, hogy a data frame 1704
megfigyelést, azaz sort és 6 változót, vagyis oszlopot tartalmaz. A
sorok leválogatásához a \texttt{dplyr} csomag \texttt{filter()}
parancsát használva, a \texttt{\%\textgreater{}\%} operátor segítségével
leválogatjuk azon országok 1962-es adatait, ahol a várható élettartam
meghaladta a 70 évet:

\begin{Shaded}
\begin{Highlighting}[]
\NormalTok{gapminder\_df }\SpecialCharTok{\%\textgreater{}\%}
  \FunctionTok{filter}\NormalTok{(year }\SpecialCharTok{==} \DecValTok{1962}\NormalTok{, lifeExp }\SpecialCharTok{\textgreater{}} \DecValTok{70}\NormalTok{)}
\end{Highlighting}
\end{Shaded}

\begin{verbatim}
## # A tibble: 16 x 6
##    country         continent  year lifeExp       pop gdpPercap
##    <fct>           <fct>     <int>   <dbl>     <int>     <dbl>
##  1 Australia       Oceania    1962    70.9  10794968    12217.
##  2 Belgium         Europe     1962    70.2   9218400    10991.
##  3 Canada          Americas   1962    71.3  18985849    13462.
##  4 Denmark         Europe     1962    72.4   4646899    13583.
##  5 France          Europe     1962    70.5  47124000    10560.
##  6 Germany         Europe     1962    70.3  73739117    12902.
##  7 Iceland         Europe     1962    73.7    182053    10350.
##  8 Ireland         Europe     1962    70.3   2830000     6632.
##  9 Netherlands     Europe     1962    73.2  11805689    12791.
## 10 New Zealand     Oceania    1962    71.2   2488550    13176.
## 11 Norway          Europe     1962    73.5   3638919    13450.
## 12 Slovak Republic Europe     1962    70.3   4237384     7481.
## 13 Sweden          Europe     1962    73.4   7561588    12329.
## 14 Switzerland     Europe     1962    71.3   5666000    20431.
## 15 United Kingdom  Europe     1962    70.8  53292000    12477.
## 16 United States   Americas   1962    70.2 186538000    16173.
\end{verbatim}

De ugyanígy leválogathatjuk a data frame-ből az adatokat egy
karakterlánc és egy logikai művelet alapján.

\begin{Shaded}
\begin{Highlighting}[]
\NormalTok{gapminder\_df }\SpecialCharTok{\%\textgreater{}\%}
  \FunctionTok{filter}\NormalTok{(country }\SpecialCharTok{==} \StringTok{"Sweden"}\NormalTok{, year }\SpecialCharTok{\textgreater{}} \DecValTok{1990}\NormalTok{)}
\end{Highlighting}
\end{Shaded}

\begin{verbatim}
## # A tibble: 4 x 6
##   country continent  year lifeExp     pop gdpPercap
##   <fct>   <fct>     <int>   <dbl>   <int>     <dbl>
## 1 Sweden  Europe     1992    78.2 8718867    23880.
## 2 Sweden  Europe     1997    79.4 8897619    25267.
## 3 Sweden  Europe     2002    80.0 8954175    29342.
## 4 Sweden  Europe     2007    80.9 9031088    33860.
\end{verbatim}

Itt tehát a data frame azon sorait szeretnénk látni, ahol az ország
megegyezik a „Sweden" karakterlánccal az év pedig nagyobb, mint 1990.

A \texttt{select()} parancs segítségével leválogathatunk oszlopokat a
data frame-ből, a \texttt{mutate()} segítségével pedig új oszlopot
adhatunk hozzá:

\begin{Shaded}
\begin{Highlighting}[]
\NormalTok{gapminder\_df }\SpecialCharTok{\%\textgreater{}\%}
  \FunctionTok{select}\NormalTok{(country, year, pop) }\SpecialCharTok{\%\textgreater{}\%} \CommentTok{\# leválogatjuk a country, year, pop oszlopokat}
  \FunctionTok{mutate}\NormalTok{(}\AttributeTok{pop\_k =}\NormalTok{ pop }\SpecialCharTok{/} \DecValTok{1000}\NormalTok{) }\CommentTok{\# létrehozzuk a pop\_k oszlopot és meghatározzuk, hogy mit tartalmazzon}
\end{Highlighting}
\end{Shaded}

\begin{verbatim}
## # A tibble: 1,704 x 4
##    country      year      pop  pop_k
##    <fct>       <int>    <int>  <dbl>
##  1 Afghanistan  1952  8425333  8425.
##  2 Afghanistan  1957  9240934  9241.
##  3 Afghanistan  1962 10267083 10267.
##  4 Afghanistan  1967 11537966 11538.
##  5 Afghanistan  1972 13079460 13079.
##  6 Afghanistan  1977 14880372 14880.
##  7 Afghanistan  1982 12881816 12882.
##  8 Afghanistan  1987 13867957 13868.
##  9 Afghanistan  1992 16317921 16318.
## 10 Afghanistan  1997 22227415 22227.
## # ... with 1,694 more rows
\end{verbatim}

\hypertarget{vizualizuxe1ciuxf3}{%
\section{Vizualizáció}\label{vizualizuxe1ciuxf3}}

Az elemzéseinkhez használt data frame adatainak alapján a
\texttt{ggplot2} csomag segítségével lehetőségünk van különböző
vizualizációk készítésére is.

A \texttt{ggplot2} használata során különböző témákat alkalmazhatunk,
melyek részletes leírása megtalálható:
\url{https://ggplot2.tidyverse.org/reference/ggtheme.html}

Abban az esetben, ha nem választunk témát, a \texttt{ggplot2} a
következő ábrán is látható alaptémát használja. Ha például a szürke
helyett fehér hátteret szeretnénk, alkalmazhatjuk a
\texttt{theme\_minmal()}parancsot. Szintén gyakran alkalmazott ábra alap
a \texttt{thema\_bw()}, ami az előzőtől az ábra keretezésében
különbözik. Ha fehér alapon, de a beosztások vonalait feketén szeretnénk
megjeleníteni, alkalmazhatjuk a \texttt{theme\_linedraw()} függvényt, a
\texttt{theme\_void()} segítségével pedig egy fehér alapon,
beosztásoktól mentes alapot kapunk, a \texttt{theme\_dark()} pedig sötét
hátteret eredményez. A \texttt{theme\_classic()} segítségével az x és y
tengelyt jeleníthetjük meg fehér alapon.

Egy ábra készítésének alapja mindig a használni kívánt adatkészlet
beolvasása, illetve az ábrázolni kiíván változtót vagy változók
megadása.

Ezt követi a megfelelő alakzat kiválasztása, attól függően például, hogy
eloszlást, változást, adatok közötti kapcsolatot, vagy elétéseket
akarunk ábrázolni. A \texttt{geom} az a geometriai objektum, a mit a
diagram az adatok megjelenítésére használ. A\texttt{gglpot2} több mint
40 féle alakzat alkalmazására ad lehetőséget, ezek közül néhány
gyakoribbat mutatunk be az alábbiakban. Az alakzatokról részletes
leírása található például az alábbi linken:
\url{https://r4ds.had.co.nz/data-visualisation.html}

A következőkben a már korábban is használt \texttt{gapminder} adatok
segítségével, személetetjük az adatok vizualizálásának alapjait. Először
egyszerű alapbeállítások mellett egy histogram típusú vizualizációt
készítünk.

\begin{Shaded}
\begin{Highlighting}[]
\FunctionTok{ggplot}\NormalTok{(}
  \AttributeTok{data =}\NormalTok{ gapminder\_df, }\CommentTok{\# itt adjuk meg az adatkészletet}
  \AttributeTok{mapping =} \FunctionTok{aes}\NormalTok{(}\AttributeTok{x =}\NormalTok{ gdpPercap)}
\NormalTok{) }\SpecialCharTok{+} \CommentTok{\# majd a változót}
  \FunctionTok{geom\_histogram}\NormalTok{() }\CommentTok{\# és az alakzatot}
\end{Highlighting}
\end{Shaded}

\includegraphics{_main_files/figure-latex/unnamed-chunk-8-1.pdf}

Lehetőségünk van arra, hogy az alakzat színét megváltoztatássuk. A
használható színek és színkódok megtalálhatóak a \texttt{ggplot2}
leírásában: \url{https://ggplot2-book.org/scale-colour.html}

\begin{Shaded}
\begin{Highlighting}[]
\FunctionTok{ggplot}\NormalTok{(}
  \AttributeTok{data =}\NormalTok{ gapminder\_df,}
  \AttributeTok{mapping =} \FunctionTok{aes}\NormalTok{(}\AttributeTok{x =}\NormalTok{ gdpPercap)}
\NormalTok{) }\SpecialCharTok{+}
  \FunctionTok{geom\_histogram}\NormalTok{(}\AttributeTok{fill =} \StringTok{"yellow"}\NormalTok{, }\AttributeTok{colour =} \StringTok{"green"}\NormalTok{) }\CommentTok{\# a fill = után idézojelben adjuk meg az alakzat kitöltésére, a colour = után pedig a körberajzolására használni kívánt színt}
\end{Highlighting}
\end{Shaded}

\includegraphics{_main_files/figure-latex/unnamed-chunk-9-1.pdf}

Meghatározhatjuk külön-külön a histogram x és y tengelyén ábrázolni
kívánt adatokat és választhatjuk azok pontszerű ábrázolását is.

\begin{Shaded}
\begin{Highlighting}[]
\FunctionTok{ggplot}\NormalTok{(}
  \AttributeTok{data =}\NormalTok{ gapminder\_df,}
  \AttributeTok{mapping =} \FunctionTok{aes}\NormalTok{(}
    \AttributeTok{x =}\NormalTok{ gdpPercap,}
    \AttributeTok{y =}\NormalTok{ lifeExp}
\NormalTok{  )}
\NormalTok{) }\SpecialCharTok{+}
  \FunctionTok{geom\_point}\NormalTok{() }\CommentTok{\# itt választjuk a pontszeru ábrázolást}
\end{Highlighting}
\end{Shaded}

\includegraphics{_main_files/figure-latex/unnamed-chunk-10-1.pdf}

Ahogy az előzőekben, itt is megváltoztathatjuk az ábra színét.

\begin{Shaded}
\begin{Highlighting}[]
\FunctionTok{ggplot}\NormalTok{(}
  \AttributeTok{data =}\NormalTok{ gapminder\_df,}
  \AttributeTok{mapping =} \FunctionTok{aes}\NormalTok{(}
    \AttributeTok{x =}\NormalTok{ gdpPercap,}
    \AttributeTok{y =}\NormalTok{ lifeExp}
\NormalTok{  )}
\NormalTok{) }\SpecialCharTok{+}
  \FunctionTok{geom\_point}\NormalTok{(}\AttributeTok{colour =} \StringTok{"blue"}\NormalTok{)}
\end{Highlighting}
\end{Shaded}

\includegraphics{_main_files/figure-latex/unnamed-chunk-11-1.pdf}

Az fenti script kibővítésével az egyes kontinensek adatait különböző
színnel ábrázolhatjuk, az x és y tengelyt elnevezhetjük, a histogramnak
címet és alcímet adhatunk, illetve az adataink forrását is
feltüntethetjük az alábbi módon:

\begin{Shaded}
\begin{Highlighting}[]
\FunctionTok{ggplot}\NormalTok{(}
  \AttributeTok{data =}\NormalTok{ gapminder\_df,}
  \AttributeTok{mapping =} \FunctionTok{aes}\NormalTok{(}
    \AttributeTok{x =}\NormalTok{ gdpPercap,}
    \AttributeTok{y =}\NormalTok{ lifeExp,}
    \AttributeTok{color =}\NormalTok{ continent}
\NormalTok{  )}
\NormalTok{) }\SpecialCharTok{+} \CommentTok{\# a kontinensek adataitkülönbözo színekkel ábrázolja}
  \FunctionTok{geom\_point}\NormalTok{() }\SpecialCharTok{+}
  \FunctionTok{labs}\NormalTok{(}
    \AttributeTok{x =} \StringTok{"GDP per capita (log $)"}\NormalTok{, }\CommentTok{\# a labs()segítségével nevezhetjük el a tengelyeket, adhatunk fo és alcímeket az ábrának}
    \AttributeTok{y =} \StringTok{"Life expectancy"}\NormalTok{,}
    \AttributeTok{title =} \StringTok{"Connection between GDP and Life expectancy"}\NormalTok{,}
    \AttributeTok{subtitle =} \StringTok{"Points are country{-}years"}\NormalTok{,}
    \AttributeTok{caption =} \StringTok{"Source: Gapminder dataset"}
\NormalTok{  )}
\end{Highlighting}
\end{Shaded}

\includegraphics{_main_files/figure-latex/unnamed-chunk-12-1.pdf}

Az ábrán található feliratok méretének, betűtípusának és betűszínének
megválasztásra is lehetőségünk van.

\begin{Shaded}
\begin{Highlighting}[]
\FunctionTok{ggplot}\NormalTok{(}
  \AttributeTok{data =}\NormalTok{ gapminder\_df,}
  \AttributeTok{mapping =} \FunctionTok{aes}\NormalTok{(}
    \AttributeTok{x =}\NormalTok{ gdpPercap,}
    \AttributeTok{y =}\NormalTok{ lifeExp,}
    \AttributeTok{color =}\NormalTok{ continent}
\NormalTok{  )}
\NormalTok{) }\SpecialCharTok{+} \CommentTok{\# a kontinensek adataitkülönbözo színekkel ábrázolja}
  \FunctionTok{geom\_point}\NormalTok{() }\SpecialCharTok{+}
  \FunctionTok{labs}\NormalTok{(}
    \AttributeTok{x =} \StringTok{"GDP per capita (log $)"}\NormalTok{, }\CommentTok{\# a labs()segítségével nevezhetjük el a tengelyeket, adhatunk fo és alcímeket az ábrának}
    \AttributeTok{y =} \StringTok{"Life expectancy"}\NormalTok{,}
    \AttributeTok{title =} \StringTok{"Connection between GDP and Life expectancy"}\NormalTok{,}
    \AttributeTok{subtitle =} \StringTok{"Points are country{-}years"}\NormalTok{,}
    \AttributeTok{caption =} \StringTok{"Source: Gapminder dataset"}
\NormalTok{  ) }\SpecialCharTok{+}
  \FunctionTok{theme}\NormalTok{(}\AttributeTok{plot.title =} \FunctionTok{element\_text}\NormalTok{(}
    \AttributeTok{size =} \DecValTok{20}\NormalTok{, }\CommentTok{\# megadhatjuk a kívánt betumértetet}
    \AttributeTok{colour =} \StringTok{"red"}\NormalTok{, }\CommentTok{\# megadhatjuk a kívánt betuszínt}
    \AttributeTok{face =} \StringTok{"italic"}\NormalTok{, }\CommentTok{\# beállíthatjuk, hogy a szöveg dolt betus legyen}
    \AttributeTok{family =} \StringTok{"Courier"}
\NormalTok{  )) }\CommentTok{\# meggadhatjuk a kívánt betutípust}
\end{Highlighting}
\end{Shaded}

\includegraphics{_main_files/figure-latex/unnamed-chunk-13-1.pdf}

Készíthetünk oszlopdiagramot is, amit a \texttt{ggplot2} diamonds
adatkészletén személtetünk

\begin{Shaded}
\begin{Highlighting}[]
\FunctionTok{ggplot}\NormalTok{(}\AttributeTok{data =}\NormalTok{ diamonds) }\SpecialCharTok{+}
  \FunctionTok{geom\_bar}\NormalTok{(}\AttributeTok{mapping =} \FunctionTok{aes}\NormalTok{(}\AttributeTok{x =}\NormalTok{ cut))}
\end{Highlighting}
\end{Shaded}

\includegraphics{_main_files/figure-latex/unnamed-chunk-14-1.pdf}

Itt is lehetőségünk van arra, hogy a diagram színét megváltoztassuk.

\begin{Shaded}
\begin{Highlighting}[]
\FunctionTok{ggplot}\NormalTok{(}\AttributeTok{data =}\NormalTok{ diamonds) }\SpecialCharTok{+}
  \FunctionTok{geom\_bar}\NormalTok{(}\AttributeTok{mapping =} \FunctionTok{aes}\NormalTok{(}\AttributeTok{x =}\NormalTok{ cut), }\AttributeTok{fill =} \StringTok{"darkgreen"}\NormalTok{)}
\end{Highlighting}
\end{Shaded}

\includegraphics{_main_files/figure-latex/unnamed-chunk-15-1.pdf}

De arra is lehetőségünk van, hogy az egyes oszlopok eltérő színűek
legyenek.

\begin{Shaded}
\begin{Highlighting}[]
\FunctionTok{ggplot}\NormalTok{(}\AttributeTok{data =}\NormalTok{ diamonds) }\SpecialCharTok{+}
  \FunctionTok{geom\_bar}\NormalTok{(}\AttributeTok{mapping =} \FunctionTok{aes}\NormalTok{(}\AttributeTok{x =}\NormalTok{ cut, }\AttributeTok{fill =}\NormalTok{ cut))}
\end{Highlighting}
\end{Shaded}

\includegraphics{_main_files/figure-latex/unnamed-chunk-16-1.pdf} Arra
is van lehetőségünk, hogy egyszerre több változót is ábrázoljunk.

\begin{Shaded}
\begin{Highlighting}[]
\FunctionTok{ggplot}\NormalTok{(}\AttributeTok{data =}\NormalTok{ diamonds) }\SpecialCharTok{+}
  \FunctionTok{geom\_bar}\NormalTok{(}\AttributeTok{mapping =} \FunctionTok{aes}\NormalTok{(}\AttributeTok{x =}\NormalTok{ cut, }\AttributeTok{fill =}\NormalTok{ clarity))}
\end{Highlighting}
\end{Shaded}

\includegraphics{_main_files/figure-latex/unnamed-chunk-17-1.pdf}

Arra ggplot2 segítségével arra is lehetőségünk van, hogy csv-ből
beolvasott adatainkat vizualizáljuk.

\begin{Shaded}
\begin{Highlighting}[]
\NormalTok{plot\_cap\_1 }\OtherTok{\textless{}{-}} \FunctionTok{read.csv}\NormalTok{(}\StringTok{"data/plot\_cap\_1.csv"}\NormalTok{, }\AttributeTok{head =} \ConstantTok{TRUE}\NormalTok{, }\AttributeTok{sep =} \StringTok{";"}\NormalTok{) }\CommentTok{\# beolvassuk a csv fájlt, megadva, hogy az egyes oszlopokat \textquotesingle{};\textquotesingle{} határolja}
\FunctionTok{ggplot}\NormalTok{(plot\_cap\_1, }\FunctionTok{aes}\NormalTok{(Year, }\AttributeTok{fill =}\NormalTok{ Subtopic)) }\SpecialCharTok{+} \CommentTok{\# megadjuk, hogy az adatokat évente és azon belül subtopic{-}ok szerint szeretnénk rendezni}
  \FunctionTok{scale\_x\_discrete}\NormalTok{(}\AttributeTok{limits =} \FunctionTok{c}\NormalTok{(}\DecValTok{1957}\NormalTok{, }\DecValTok{1958}\NormalTok{, }\DecValTok{1959}\NormalTok{, }\DecValTok{1960}\NormalTok{, }\DecValTok{1961}\NormalTok{, }\DecValTok{1962}\NormalTok{, }\DecValTok{1963}\NormalTok{)) }\SpecialCharTok{+}
  \FunctionTok{geom\_bar}\NormalTok{(}\AttributeTok{position =} \StringTok{"dodge"}\NormalTok{) }\SpecialCharTok{+} \CommentTok{\# meghatározzuk az x tengely értékeit}
  \FunctionTok{labs}\NormalTok{(}
    \AttributeTok{x =} \ConstantTok{NULL}\NormalTok{, }\AttributeTok{y =} \ConstantTok{NULL}\NormalTok{, }\CommentTok{\# az x és az y tengely nem kap külön feliratot}
    \AttributeTok{title =} \StringTok{"A Magyar Közlönyben kihirdetett agrárpolitikai jogszabályok"}\NormalTok{, }\CommentTok{\# az ábra címe}
    \AttributeTok{subtitle =} \StringTok{"N=445"}
\NormalTok{  ) }\SpecialCharTok{+} \CommentTok{\# az ábra alcíme}
  \FunctionTok{coord\_flip}\NormalTok{() }\SpecialCharTok{+} \CommentTok{\# az ábra tipusa}
  \FunctionTok{theme\_minimal}\NormalTok{() }\SpecialCharTok{+}
  \FunctionTok{theme}\NormalTok{(}\AttributeTok{plot.title =} \FunctionTok{element\_text}\NormalTok{(}\AttributeTok{size =} \DecValTok{12}\NormalTok{)) }\CommentTok{\# az ábra címének betumérete}
\end{Highlighting}
\end{Shaded}

A csv-ből belolvasott adatainból kördiagramot is készíthetünk

\begin{Shaded}
\begin{Highlighting}[]
\NormalTok{pie }\OtherTok{\textless{}{-}} \FunctionTok{read.csv}\NormalTok{(}\StringTok{"data/pie.csv"}\NormalTok{, }\AttributeTok{head =} \ConstantTok{TRUE}\NormalTok{, }\AttributeTok{sep =} \StringTok{";"}\NormalTok{)}

\FunctionTok{ggplot}\NormalTok{(pie, }\FunctionTok{aes}\NormalTok{(}\AttributeTok{x =} \StringTok{""}\NormalTok{, }\AttributeTok{y =}\NormalTok{ value, }\AttributeTok{fill =}\NormalTok{ Type)) }\SpecialCharTok{+}
  \FunctionTok{geom\_bar}\NormalTok{(}\AttributeTok{stat =} \StringTok{"identity"}\NormalTok{, }\AttributeTok{width =} \DecValTok{1}\NormalTok{) }\SpecialCharTok{+}
  \FunctionTok{coord\_polar}\NormalTok{(}\StringTok{"y"}\NormalTok{, }\AttributeTok{start =} \DecValTok{0}\NormalTok{) }\SpecialCharTok{+}
  \FunctionTok{scale\_fill\_brewer}\NormalTok{(}\AttributeTok{palette =} \StringTok{"GnBu"}\NormalTok{) }\SpecialCharTok{+}
  \FunctionTok{labs}\NormalTok{(}
    \AttributeTok{title =} \StringTok{"A Magyar Közlönyben megjelent jogszabályok típusai"}\NormalTok{,}
    \AttributeTok{subtitle =} \StringTok{"N = 445"}
\NormalTok{  ) }\SpecialCharTok{+}
  \FunctionTok{theme\_void}\NormalTok{()}
\end{Highlighting}
\end{Shaded}

\hypertarget{leuxedruxf3-statisztika-szuxf3zsuxe1k-uxe9s-szuxf3eloszluxe1sok}{%
\chapter{Leíró statisztika: szózsák és
szóeloszlások}\label{leuxedruxf3-statisztika-szuxf3zsuxe1k-uxe9s-szuxf3eloszluxe1sok}}

negyedik fejezet

\hypertarget{a-szuxf6vegek-reprezentuxe1luxe1sa-a-vektortuxe9rben}{%
\chapter{A szövegek reprezentálása a
vektortérben}\label{a-szuxf6vegek-reprezentuxe1luxe1sa-a-vektortuxe9rben}}

otodik fejezet

\hypertarget{a-korpuszuxe9puxedtuxe9s-probluxe9muxe1i-uxe9s-a-szuxf6vegelux151kuxe9szuxedtuxe9s}{%
\chapter{A korpuszépítés problémái és a
szövegelőkészítés}\label{a-korpuszuxe9puxedtuxe9s-probluxe9muxe1i-uxe9s-a-szuxf6vegelux151kuxe9szuxedtuxe9s}}

hatodik fejezet

\hypertarget{szuxf3tuxe1ralapuxfa-elemzuxe9sek-uxe9rzelem-elemzuxe9s}{%
\chapter{Szótáralapú elemzések,
érzelem-elemzés}\label{szuxf3tuxe1ralapuxfa-elemzuxe9sek-uxe9rzelem-elemzuxe9s}}

hetedik fejezet

\hypertarget{klaszterelemzuxe9s-uxe9s-topic-modellezuxe9s}{%
\chapter{Klaszterelemzés és topic
modellezés}\label{klaszterelemzuxe9s-uxe9s-topic-modellezuxe9s}}

nyolcadik fejezet

\hypertarget{szuxf3beuxe1gyazuxe1sok}{%
\chapter{Szóbeágyazások}\label{szuxf3beuxe1gyazuxe1sok}}

kilencedik fejezet

\hypertarget{szuxf6veguxf6sszehasonluxedtuxe1s}{%
\chapter{Szövegösszehasonlítás}\label{szuxf6veguxf6sszehasonluxedtuxe1s}}

tizedik fejezet

\hypertarget{termuxe9szetes-nyelv-feldolgozuxe1s-nlp}{%
\chapter{Természetes-nyelv feldolgozás
(NLP)}\label{termuxe9szetes-nyelv-feldolgozuxe1s-nlp}}

tizenegyedik fejezet

\hypertarget{osztuxe1lyozuxe1s-uxe9s-feluxfcgyelt-tanuluxe1s}{%
\chapter{Osztályozás és felügyelt
tanulás}\label{osztuxe1lyozuxe1s-uxe9s-feluxfcgyelt-tanuluxe1s}}

tizenkeddik fejezet

\hypertarget{fuxfcggeluxe9k}{%
\chapter{Függelék}\label{fuxfcggeluxe9k}}

\hypertarget{az-r-uxe9s-az-rstudio-hasznuxe1lata}{%
\section{Az R és az RStudio
használata}\label{az-r-uxe9s-az-rstudio-hasznuxe1lata}}

Az R egy programozási nyelv, amely alkalmas statisztikai számítások
elvégzésére és ezek eredményeinek grafikus megjelenítésére. Az R
ingyenes, nyílt forráskódú szoftver, mely telepíthető mind Windows, mind
Linux, mind MacOS operációs rendszerek alatt, az alábbi oldalról:
\url{https://cran.r-project.org/} Az RStudio az R integrált fejlesztői
környezete (\emph{integrated development environment, IDE}), mely egy
olyan felhasználóbarát felületet biztosít, ami egyszerűbb és átláthatóbb
munkát tesz lehetővé. Az RStudio az alábbi oldalról tölthető le:
\url{https://rstudio.com/products/rstudio/download/}

A „point and click" szoftverekkel szemben az R használata során kódot
kell írni, ami bizonyos programozási jártasságot feltételez, de a
későbbiekben lehetővé teszi azt adott kutatási kérdéshez maximálisan
illeszkedő kódok összeállítását, melyek segítségével az elemzések mások
számára is megbízhatóan reprodukálhatóak lesznek. Ugyancsak az R
használata mellett szól, hogy komoly fejlesztői és felhasználói
közösséggel rendelkezik, így a használat során felmerülő problémákra
általában gyorsan megoldást találhatunk.

\hypertarget{az-rstudio-kezdux151feluxfclete}{%
\subsection{Az RStudio
kezdőfelülete}\label{az-rstudio-kezdux151feluxfclete}}

Az RStudio kezdőfelülete négy panelből, eszközsorból és menüsorból áll:

\begin{figure}

{\centering \includegraphics{figures/13-01_layout} 

}

\caption{RStudio felhasználói felület}\label{fig:unnamed-chunk-20}
\end{figure}

Az \textbf{\emph{(1) editor}} ablak szolgál a kód beírására, futtatására
és mentésére. A \textbf{\emph{(2) console}} ablakban jelenik meg a
lefuttatott kód és az eredmények. A jobb felső ablak \textbf{\emph{(3)
environment}} fülén láthatóak a memóriában tárolt adatállományok,
változók és felhasználói függvények. A \textbf{\emph{history}} fül
mutatja a korábban lefuttatott utasításokat. A jobb alsó ablak
\textbf{\emph{(4) files}} fülén az aktuális munkakönyvtárban levő
mappákat és fájlok találjuk, míg a \textbf{\emph{plot}} fülön az
elemzéseink során elkészített ábrák jelennek meg. A
\textbf{\emph{packages}} fülön frissíthetjük a meglévő r csomagokat és
telepíthetünk újakat. A \textbf{\emph{help}} fülön a különböző
függvények, parancsok leírását, és használatát találjuk meg. A
\texttt{Tools\ -\textgreater{}\ Global\ Options} menüpont végezhetjük el
az RStudio testreszabását. Így például beállíthatjuk az ablaktér
elrendezését (\emph{Pane layout}), vagy a színvilágot
(\emph{Appearance}), illetve azt hogy a kódok ne fussanak ki az ablakból
(\texttt{Code\ -\textgreater{}\ Editing\ -\textgreater{}\ Soft\ wrap\ R\ source\ files})

\hypertarget{projekt-alapuxfa-munka}{%
\subsection{Projekt alapú munka}\label{projekt-alapuxfa-munka}}

Bár nem kötelező, de javasolt, hogy az RStudio-ban projekt alapon
dolgozzunk, mivel így az összes -- az adott projekttel kapcsolatos fájlt
-- egy mappában tárolhatjuk. Új projekt beállítását a
\texttt{File-\textgreater{}New\ Project} menüben tehetjük meg, ahol a
saját gépünk egy könyvtárát kell kiválasztani, ahová az R scripteket, az
adat- és előzményfájlokat menti. Ezenkívül a
\texttt{Tools-\textgreater{}Global\ Options-\textgreater{}General}
menüpont alatt le kell tiltani a \emph{„Restore most recently opened
project at startup''} és a \emph{„Restore .RData ino workspace at
startup''} beállítást, valamint \emph{„Save workspace to .RData on
exit''} értékre be kell állítani a \emph{„Never''} értéket.

\begin{figure}

{\centering \includegraphics{figures/13-02_project_options} 

}

\caption{RStudio projekt beállítások}\label{fig:unnamed-chunk-21}
\end{figure}

A szükséges beállítások után a
\texttt{File\ -\textgreater{}\ New\ Project} menüben hozhatjuk létre a
projektet. Itt arra is lehetőségünk van, hogy kiválasszuk, hogy a
projektünket egy teljesen új könyvtárba, vagy egy meglévőbe kívánjuk
menteni, esetleg egy meglévő projekt új verzióját szeretnénk létrehozni.
Ha sikeresen létrehoztuk a projektet, az RStudio jobb felső sarkában
látnunk kell annak nevét.

\hypertarget{scriptek-szerkesztuxe9se-fuxfcggvuxe9nyek-hasznuxe1lata}{%
\subsection{Scriptek szerkesztése, függvények
használata}\label{scriptek-szerkesztuxe9se-fuxfcggvuxe9nyek-hasznuxe1lata}}

Új script a
\texttt{File\ -\textgreater{}\ New\ -\textgreater{}\ File\ -\textgreater{}\ R}
Script menüpontban hozható létre, mentésére a File-\textgreater Save
menüpontban egy korábbi script megnyitására
\texttt{File\ -\textgreater{}\ Open} menüpontban van lehetőségünk.
Script bármilyen szövegszerkesztővel írható és beilleszthető az editor
ablakba. A scripteket érdemes magyarázatokkal (kommentekkel) ellátni,
hogy a későbbiekben pontosan követhető legyen, hogy melyik parancs
segítségével pontosan milyen lépéseket hajtottunk végre. A
magyarázatokat vagy más néven kommenteket kettőskereszt (\texttt{\#})
karakterrel vezetjük be. A scriptbeli utasítások az azokat tartalmazó
sorokra állva vagy több sort kijelölve a Run feliratra kattintva vagy a
\texttt{Ctrl+Enter} billentyűparanccsal futtathatók le. A lefuttatott
parancsok és azok eredményei ezután a bal alsó sarokban lévő console
ablakban jelennek meg és ugyanitt kapunk hibaüzenetet is, ha valamilyen
hibát vétettünk a scriptben.

A munkafolyamat során létrehozott állományok (ábrák, fájlok) ebbe az ún.
munkakönyvtárba (\emph{working directory}) mentődnek. Az aktuális
munkakönyvtár neve, elérési útja a \texttt{getwd()} utasítással
jeleníthető meg. A könyvtárban található állományok listázására a
\texttt{list.files()} utasítással van lehetőségünk. Ha a korábbiaktól
eltérő munkakönyvtárat akarunk megadni, azt a \texttt{setwd()}
függvénnyel tehetjük meg, ahol a ()-ben az adott mappa elérési útját
kell megadnunk. Az elérési útban a meghajtó azonosítóját, majd a mappák,
almappák nevét vagy egy normál irányú perjel (\texttt{/}), vagy két
fordított perjel (\texttt{\textbackslash{}\textbackslash{}}) választja
el, mivel az elérési út karakterlánc, ezért azt idézőjelek vagy
aposztrófok közé kell tennünk. Az aktuális munkakönyvtárba beléphetünk a
jobb alsó ablak file lapján a
\texttt{„More\ -\textgreater{}\ Go\ To\ Working\ Directory”}
segítségével. Ugyanitt a \texttt{„Set\ Working\ Directory”}-val
munkakönyvtárnak állíthatjuk be az a mappát, amelyben épp benne vagyunk.

\begin{figure}

{\centering \includegraphics{figures/13-03_working_directory} 

}

\caption{Working directory beállítások}\label{fig:unnamed-chunk-22}
\end{figure}

A munkafolyamat befejezésére a \texttt{q()} vagy \texttt{quit()}
függvényel van lehetőségünk. A munkafolyamat során különböző
objektumokat hozunk létre, melyek az RStudio jobb felső ablakának
environment fülén jelennek meg, a mentett objektumokat a fent látható
seprű ikonra kattintva törölhetjük a memóriából. Az environment ablakra
érdemes úgy gondolni hogy ott jelennek meg a memóriában tárolt értékek.
Az R-ben objektumokkal dolgozunk, amik a teljesség igénye nélkül
lehetnek egyszerű szám vektortok, vagy akár komplex listák, illetve
függvények, ábrák.

Az RStudio jobb alsó ablakának plots fülén láthatjuk azon parancsok
eredményét, melyek kimenete valamilyen ábra. A packages fülnél a már
telepített és a letölthető kiegészítő csomagokat jeleníthetjük meg. A
help fülön a korábban említettek szerint a súgó érhető el. Az
RStudio-ban használható billentyűparancsok teljes listáját Alt+Shift+K
billentyűkombinációval tekinthetjük meg. Néhány gyakrabban használt,
hasznos billentyűparancs:

\begin{itemize}
\tightlist
\item
  \texttt{Ctrl+Enter}: futtassa a kódot az aktuális sorban
\item
  \texttt{Ctrl+Alt+B}: futtassa a kódot az elejétől az aktuális sorig
\item
  \texttt{Ctrl+Alt+E}: futtassa a kódot az aktuális sortól a forrásfájl
  végéig
\item
  \texttt{Ctrl+D}: törölje az aktuális sort
\end{itemize}

Az R-ben beépített \textbf{függvények (function)} állnak
rendelkezésünkre a számítások végrehajtására, emellett több
\textbf{csomag (package)} is letölthető, amelyek különböző függvényeket
tartalmaznak. A függvények a következőképpen épülnek fel:
\texttt{függvénynév(paraméter)}. Például tartalom képernyőre való
kiíratását a \texttt{print()} függvénnyel tehetjük, amelynek gömbölyű
zárójelekkel határolt részébe írhatjuk a megjelenítendő szöveget. A
\texttt{citation()} függvénnyel lekérdezhetjük az egyes beépített
csomagokra való hivatkozást is: a \texttt{citation(quanteda)} függvény a
quanteda csomag hivatkozását adja meg. Az R súgórendszere a
\texttt{help.start()} utasítással indítható el. Egy adott függvényre
vonatkozó súgórészlet a függvények neve elé kérdőjel írásával, vagy a
\texttt{help()} argumentumába a kérdéses függvény nevének beírásával
jeleníthető meg (pl.: \texttt{help(sum)}).

\hypertarget{r-csomagok}{%
\subsection{R csomagok}\label{r-csomagok}}

Az R-ben telepíthetők kiegészítő csomagok (packages), amelyek
alapértelmezetten el nem érhető algoritmusokat, függvényeket
tartalmaznak. A csomagok saját dokumentációval rendelkeznek, amelyeket
fel kell tüntetni a használatukkal készült publikációink
hivatkozáslistájában. A csomagok telepítésre több lehetőségünk is van:
használhatjuk a menüsor
\texttt{Tools\ -\textgreater{}\ Install\ Packages} menüpontját, vagy a
jobb alsó ablak \emph{Packages} fül Install menüpontját, illetve az
editor ablakban az \texttt{install.packages()} parancsot futtatva, ahol
a ()-be a telepíteni kívánt csomag nevét kell beírnunk (pl.:
\texttt{install.packages(dplyr)}).

\begin{figure}

{\centering \includegraphics{figures/13-04_packages} 

}

\caption{Packages fül}\label{fig:unnamed-chunk-23}
\end{figure}

\hypertarget{objektumok-tuxe1roluxe1sa-uxe9rtuxe9kaduxe1s}{%
\subsection{Objektumok tárolása,
értékadás}\label{objektumok-tuxe1roluxe1sa-uxe9rtuxe9kaduxe1s}}

Az objektumok lehetnek például \emph{vektorok}, \emph{mátrixok}
(matrix), \emph{tömbök} (array), \emph{adat táblák} (data frame).
Értékadás nélkül az R csak megjeleníti a műveletek eredményét, de nem
tárolja el azokat. Az eredmények eltárolásához azokat egy objektumba
kell elmentenünk. Ehhez meg kell adnunk az objektum nevét majd az
\texttt{\textless{}-} után adjuk meg annak értékét:
\texttt{a\ \textless{}-\ 12\ +\ 3}.Futtatás után az environments fülön
megjelenik az a objektum, melynek értéke \texttt{15}. Az objektumok
elnevezésénél figyelnünk kell arra, hogy az R különbséget tesz a kis és
nagybetűk között, valamint, hogy az ugyanolyan nevű objektumokat kérdés
nélkül felülírja és ezt a felülírást nem lehet visszavonni.

\hypertarget{vektorok}{%
\subsection{Vektorok}\label{vektorok}}

Az R-ben kétféle típusú vektort különböztetünk meg:

\begin{itemize}
\tightlist
\item
  egyedüli vektor (atomic vector)
\item
  lista (list)
\end{itemize}

Az egyedüli vektornak hat típusa van, \textbf{logikai} (logical),
\textbf{egész szám} (integer), \textbf{természetes szám} (double),
\textbf{karakter} (character), \textbf{komplex szám} (complex) és
\textbf{nyers adat} (raw). A leggyakrabban valamilyen numerikus, logikai
vagy karakter vektorral használjuk. Az egyedüli vektorok onnan kapták a
nevüket hogy csak egy féle adattípust tudnak tárolni. A listák ezzel
szemben gyakorlatilag bármit tudnak tárolni, akár több listát is
egybeágyazhatunk.

A vektorok és listák azok az építőelemek amikből felépülnek az R
objektumaink. Több érték vagy azonos típusú objektum összefűzését a
\texttt{c()} függvénnyel végezhetjük el. A lenti példában három
különböző objektumot kreálunk, egy numerikusat, egy karaktert és egy
logikait. A karakter vektorban az elemeket időzőjellel és vesszővel
szeparáljuk. A logikai vektor csak \texttt{TRUE}, illetve \texttt{FALSE}
értékeket tartalmazhat.

\begin{Shaded}
\begin{Highlighting}[]
\NormalTok{numerikus }\OtherTok{\textless{}{-}} \FunctionTok{c}\NormalTok{(}\DecValTok{1}\NormalTok{,}\DecValTok{2}\NormalTok{,}\DecValTok{3}\NormalTok{,}\DecValTok{4}\NormalTok{,}\DecValTok{5}\NormalTok{)}

\NormalTok{karakter }\OtherTok{\textless{}{-}} \FunctionTok{c}\NormalTok{(}\StringTok{"kutya"}\NormalTok{,}\StringTok{"macska"}\NormalTok{,}\StringTok{"ló"}\NormalTok{)}

\NormalTok{logikai }\OtherTok{\textless{}{-}} \FunctionTok{c}\NormalTok{(}\ConstantTok{TRUE}\NormalTok{, }\ConstantTok{TRUE}\NormalTok{, }\ConstantTok{FALSE}\NormalTok{)}
\end{Highlighting}
\end{Shaded}

A létrehozott vektorokkal különböző műveleteket végezhetünk el, például
összeadhatjuk numerikus vektorainkat. Ebben az esetben az első vektor
első eleme a második vektor első eleméhez adódik.

\begin{Shaded}
\begin{Highlighting}[]
\FunctionTok{c}\NormalTok{(}\DecValTok{1}\SpecialCharTok{:}\DecValTok{4}\NormalTok{) }\SpecialCharTok{+} \FunctionTok{c}\NormalTok{(}\DecValTok{10}\NormalTok{,}\DecValTok{20}\NormalTok{,}\DecValTok{30}\NormalTok{,}\DecValTok{40}\NormalTok{)}
\end{Highlighting}
\end{Shaded}

\begin{verbatim}
## [1] 11 22 33 44
\end{verbatim}

A karaktervektorokat összefűzhetjük egymással. Itt egy új objektumot is
létrehoztunk, a jobb felső ablakban, az environment fülön láthatjuk,
hogy a létrejött karakter\_kombinalt objektum egy négy elemű
(hosszúságú) karaktervektor (\texttt{chr\ {[}1:4{]}}), melynek elemei a
\texttt{"kutya","macska","ló","nyúl"}. Az objektumként tárolt vektorok
tartalmát a lefuttatva írathatjuk ki a console ablakba. Habár van
\texttt{print()} függvény az R-ben, azt ilyenkor nem szükséges
használni.

\begin{Shaded}
\begin{Highlighting}[]
\NormalTok{karakter1 }\OtherTok{\textless{}{-}} \FunctionTok{c}\NormalTok{(}\StringTok{"kutya"}\NormalTok{,}\StringTok{"macska"}\NormalTok{,}\StringTok{"ló"}\NormalTok{)}
\NormalTok{karakter2 }\OtherTok{\textless{}{-}}\FunctionTok{c}\NormalTok{(}\StringTok{"nyúl"}\NormalTok{)}

\NormalTok{karakter\_kombinalt }\OtherTok{\textless{}{-}}\FunctionTok{c}\NormalTok{(karakter1, karakter2)}

\NormalTok{karakter\_kombinalt}
\end{Highlighting}
\end{Shaded}

\begin{verbatim}
## [1] "kutya"  "macska" "ló"     "nyúl"
\end{verbatim}

Ha egy vektorról szeretnénk megtudni, hogy milyen típusú azt a
\texttt{typeof()} vagy a \texttt{class()} paranccsal tehetjük meg, ahol
()-ben az adott objektumként tárolt vektor nevét kell megadnunk:
\texttt{typeof(karakter1)}. A vektor hosszúságát (benne tárolt elemek
száma vektorok esetén) a \texttt{lenght()} függvénnyel tudhatjuk meg.

\begin{Shaded}
\begin{Highlighting}[]
\FunctionTok{typeof}\NormalTok{(karakter1)}
\end{Highlighting}
\end{Shaded}

\begin{verbatim}
## [1] "character"
\end{verbatim}

\begin{Shaded}
\begin{Highlighting}[]
\FunctionTok{length}\NormalTok{(karakter1)}
\end{Highlighting}
\end{Shaded}

\begin{verbatim}
## [1] 3
\end{verbatim}

\hypertarget{faktorok}{%
\subsection{Faktorok}\label{faktorok}}

A faktorok a kategórikus adatok tárolására szolgálnak. Faktor típusú
változó a \texttt{factor()} függvénnyel hozható létre. A faktor
szintjeit (igen, semleges, nem), a \texttt{levels()} függvénnyel
kaphatjuk meg míg az adatok címkéit (tehát a kapott válaszok száma), a
\texttt{labels()} paranccsal érhetjük el.

\begin{Shaded}
\begin{Highlighting}[]
\NormalTok{survey\_response }\OtherTok{\textless{}{-}} \FunctionTok{factor}\NormalTok{(}\FunctionTok{c}\NormalTok{(}\StringTok{"igen"}\NormalTok{, }\StringTok{"semleges"}\NormalTok{, }\StringTok{"nem"}\NormalTok{, }\StringTok{"semleges"}\NormalTok{, }\StringTok{"nem"}\NormalTok{, }\StringTok{"nem"}\NormalTok{, }\StringTok{"igen"}\NormalTok{), }\AttributeTok{ordered =} \ConstantTok{TRUE}\NormalTok{)}


\FunctionTok{levels}\NormalTok{(survey\_response)}
\end{Highlighting}
\end{Shaded}

\begin{verbatim}
## [1] "igen"     "nem"      "semleges"
\end{verbatim}

\begin{Shaded}
\begin{Highlighting}[]
\FunctionTok{labels}\NormalTok{(survey\_response)}
\end{Highlighting}
\end{Shaded}

\begin{verbatim}
## [1] "1" "2" "3" "4" "5" "6" "7"
\end{verbatim}

\hypertarget{data-frame}{%
\subsection{Data frame}\label{data-frame}}

Az adat táblák (data frame) a statisztikai és adatelemzési folyamatok
egyik leggyakrabban használt adattárolási formája. Amikor lehetséges
akkor a `hosszú' formátumban használjuk (az R közösség a `tidy' jelzővel
illeti), aholtéglalap alakú adatszerkezetek, ahol minden sor egy
megfigyelés és minden oszlop egy változó {[}TIDY CITATION{]}. Egy data
frame többféle típusú adatot tartalmazhat. A data frame-k különféle
oszlopokból állhatnak, amelyek különféle típusú adatokat
tartalmazhatnak, de egy oszlop csak egy típusú adatból állhat. A lent
bemutatott data frame 7 megfigyelést és 4 féle változót tartalmaz (id,
country, pop, continent).

\begin{verbatim}
##   id      orszag nepesseg     kontinens
## 1  1    Thailand     68.7          Asia
## 2  2      Norway      5.2        Europe
## 3  3 North Korea     24.0          Asia
## 4  4      Canada     47.8 North America
## 5  5    Slovenia      2.0        Europe
## 6  6      France     63.6        Europe
## 7  7   Venezuela     31.6 South America
\end{verbatim}

A data frame-be rendezett adatokhoz különböző módon férhetünk hozzá,
például a data frame nevének majd {[}{]}-ben a kívánt sor megadásával,
kiírathatjuk a console ablakba annak tetszőleges sorát ás oszlopát:
\texttt{orszag\_adatok{[}1,\ 1{]}}. Az R több különböző módot kínál a
data frame sorainak és oszlopainak eléréséhez. A \texttt{{[}} általános
használata: \texttt{data\_frame{[}sor,\ oszlop{]}}. Egy másik megoldás a
\texttt{\$} haszálata: \texttt{data\_frame\$oszlop}.

\begin{Shaded}
\begin{Highlighting}[]
\NormalTok{orszag\_adatok[}\DecValTok{1}\NormalTok{, }\DecValTok{4}\NormalTok{]}
\end{Highlighting}
\end{Shaded}

\begin{verbatim}
## [1] Asia
## Levels: Asia Europe North America South America
\end{verbatim}

\begin{Shaded}
\begin{Highlighting}[]
\NormalTok{orszag\_adatok}\SpecialCharTok{$}\NormalTok{orszag}
\end{Highlighting}
\end{Shaded}

\begin{verbatim}
## [1] "Thailand"    "Norway"      "North Korea" "Canada"      "Slovenia"   
## [6] "France"      "Venezuela"
\end{verbatim}

\backmatter
\end{document}
